\section{随机变量的数字特征}

\subsection{数学期望}
\begin{definition}[数学期望]
    设离散型随机变量$X$的分布律为$P\{X=x_k\},k=1,2,\cdots.$若级数$\displaystyle{\sum_{k=1}^\infty x_kp_k}$
    绝对收敛,则称级数$\displaystyle{\sum_{k=1}^\infty x_kp_k}$的和为随机变量$X$的{\heiti 数学期望},记为$E(X)$
    ,即$$E(X)=\sum_{k=1}^\infty x_kp_k$$

    设连续型随机变量$X$的概率密度为$f(x)$,若积分$\displaystyle{\int _{-\infty}^\infty xf(x) \,dx}$绝对收敛,则称积分
    $\displaystyle{\int _{-\infty}^\infty xf(x) \,dx}$的值为随机变量$X$的{\heiti 数学期望},记为$E(X)$,即
    $$E(X)=\int _{-\infty}^\infty xf(x) \,dx$$
    
    数学期望$E(X)$完全由随机变量$X$的概率分布所确定。
\end{definition}

\begin{theorem}
    设$Y$是随机变量$X$的函数:$Y=g(X)$($g$是连续函数)
    \begin{enumerate}[(i)]
        \item 如果$X$是离散型随机变量,它的分布律为$P=\{X=x_k\}=p_k,k=1,2,\cdots,$若$\displaystyle{\sum _{k=1}^\infty g(x_k)p_k}$绝对收敛,
        则有 $$E(Y)=E[g(x)]=\sum _{k=1}^\infty g(x_k)p_k$$
        \item 如果$X$是连续型随机变量,它的概率密度为$f(x)$,若$\displaystyle{\int _{-\infty}^\infty g(x)f(x)\,dx}$绝对收敛,则有
        $$E(Y)=E[g(x)]=\int _{-\infty}^\infty g(x)f(x)\,dx$$
    \end{enumerate}
    
    上述定理还可以推广到两个或两个以上随机变量的函数的情况。

    例如设$Z$是随机变量$X,Y$的函数$Z=g(X,Y)$($g$是连续函数),那么,$Z$是一个一维随机变量。若二维随机变量$(X,Y)$的概率密度为$f(x,y)$,则有
    $$E(Z)=E[g(X,Y)]=\int _{-\infty}^\infty\int _{-\infty}^\infty g(x,y)f(x,y)\,dxdy$$
    这里设上式右边的积分绝对收敛。又若$(X,Y)$为离散型随机变量,其分布律为$P=\{X=x_i,Y=y_j\}=p_{ij},i,j=1,2,\cdots,$则有
    $$E(Z)=E[g(X,Y)]=\sum _{j=1}^\infty \sum_{i=1}^\infty g(x_i,y_j)p_{ij}$$
    这里设上式右边的级数绝对收敛。
\end{theorem}

\begin{theorem}
    数学期望的性质:
    \begin{enumerate}[$1^\circ$]
        \item 设$C$是常数,则有$E(C)=C$
        \item 设$X$是一个随机变量,$C$是常数,则有$$E(CX)=CE(X)$$
        \item 设$X,Y$是两个随机变量,则有$$E(X+Y)=E(X)+E(Y)$$这一性质可以推广到任意有限个随机变量之和的情况
        \item 设$X,Y$是两个相互独立的随机变量,则有$$E(XY)=E(X)E(Y)$$这一性质可以推广到任意有限个相互独立的随机变量之积的情况
    \end{enumerate}
\end{theorem}

\subsection{方差}
\begin{definition}[方差]
    设$X$是一个随机变量,若$E\{{[X-E(X)]}^2\}$存在,则称$E\{{[X-E(X)]}^2\}$为$X$的方差,记为$D(X)$或$\mathrm{Var}(X)$,即
    $$D(X)=\mathrm{Var}(X)=E\{{[X-E(X)]}^2\}$$

    在应用上还引入量$\sqrt{D(X)}$,记为$\sigma(X)$,称为{\heiti 标准差}或{\heiti 均方差}。

    对于离散型随机变量,有$$D(X)=\sum _{k=1}^\infty {[x_k-E(X)]}^2p_k$$
    其中$P=\{X=x_k\}=p_k,k=1,2,\cdots$是$X$的分布律

    对于连续型随机变量,有$$D(X)=\int _{-\infty}^\infty {[x-E(X)]}^2f(x)\,dx$$
    其中$f(x)$是$X$的概率密度

    随机变量$X$的方差可按下列公式计算$$D(X)=E(X^2)-{[E(X)]}^2$$
\end{definition}

\begin{theorem}
    方差的性质:
    \begin{enumerate}[$1^\circ$]
        \item 设$C$是常数,则$D(C)=0$
        \item 设$X$是随机变量,$C$是常数,则有$$D(CX)=C^2D(X)\qquad  D(X+C)=D(X)$$
        \item 设$X,Y$是两个随机变量,则有$$D(X+Y)=D(X)+D(Y)+2E\{[X-E(X)][Y-E(Y)]\}$$特别地,若$X,Y$相互独立,则有$$D(X+Y)=D(X)+D(Y)$$这一性质可以推广到任意有限多个相互独立的随机变量之和的情况
        \item $D(X)=0$的充要条件是$X$以概率$1$取常数$E(X)$,即$$P\{X=E(X)\}=1$$
    \end{enumerate}
\end{theorem}

\begin{theorem}[切比雪夫(Chebyshev)不等式]
    设随机变量$X$具有数学期望$E(X)=\mu$,方差$D(X)=\sigma^2$,则对于任意整数$\varepsilon$,有不等式
    $$P\{|X-\mu|\geq \varepsilon\}\leq \frac{\sigma^2}{\varepsilon^2}$$
    也可写为$$P\{|X-\mu|<\varepsilon\}\geq 1- \frac{\sigma^2}{\varepsilon^2}$$
\end{theorem}

\subsection{协方差及相关系数}
\begin{definition}[协方差、相关系数]
    量$E\{[X-E(X)][Y-E(Y)]\}$称为随机变量$X$与$Y$的{\heiti 协方差},记为$\mathrm{Cov}(X,Y)$,即
    $$\mathrm{Cov}(X,Y)=E\{[X-E(X)][Y-E(Y)]\}$$
    而$$\rho _{XY}=\frac{\mathrm{Cov}(X,Y)}{\sqrt{D(X)}\sqrt{D(Y)}}$$
    称为随机变量的$X$和$Y$的{\heiti 相关系数}。

    协方差的计算公式:
    $$D(X+Y)=D(X)+D(Y)+2\mathrm{Cov}(X,Y) \qquad \mathrm{Cov}(X,Y)=E(XY)-E(X)E(Y)$$
\end{definition}

\begin{theorem}
    协方差的性质:
    \begin{enumerate}
        \item $\mathrm{Cov}(aX,bY)=ab\mathrm{Cov}(X,Y)$,$a,b$是常数;
        \item $\mathrm{Cov}(X_1+X_2,Y)=\mathrm{Cov}(X_1,Y)+\mathrm{Cov}(X_2,Y)$
    \end{enumerate}
\end{theorem}

\begin{theorem}
    $\rho_{XY}$的性质:
    \begin{enumerate}
        \item $|\rho_{XY}|\leq 1$
        \item $|\rho_{XY}|= 1$的充要条件是,存在常数$a,b$,使得$P\{Y=a+bX\}=1$
    \end{enumerate}

    $\rho_{XY}$是一个可以用来表征$X,Y$之间线性关系紧密程度的量,当$|\rho_{XY}|$较大时,$X,Y$线性相关程度较好,当$|\rho_{XY}|$较小时,$X,Y$线性相关程度较差,
    当$|\rho_{XY}|=0$时,称$X$和$Y$不相关。    
\end{theorem}

\subsection{矩、协方差矩阵}
\begin{definition}
    设$X$和$Y$是随机变量。\\若
    $$E(X^k) \quad k=1,2,\cdots$$
    存在,则称它为$X$的{\heiti $k$阶原点矩},简称{\heiti $k$阶矩}.
    \\若$$E\{[X-E(X)]^k\} \quad k=2,3,\cdots$$
    存在,则称它为$X$的{\heiti $k$阶中心矩}.
    \\若$$E\{[X-E(X)]^k[Y-E(Y)]^l\} \quad k,l=1,2,\cdots$$
    存在,则称它为$X$和$Y$的{\heiti $k+l$阶混合中心矩}.
\end{definition}

\begin{definition}[协方差矩阵]
    先以二维随机变量为例。二维随机变量$(X_1,X_2)$有四个二阶中心矩(设它们都存在),分别记为
    $$c_{11}=E\{{[X_1-E(X_1)]}^2\}$$
    $$c_{12}=E\{[X_1-E(X_1)][X_2-E(X_2)]\}$$
    $$c_{21}=E\{[X_2-E(X_2)][X_1-E(X_1)]\}$$
    $$c_{22}=E\{{[X_2-E(X_2)]}^2\}$$
    将它们排成矩阵的形式
    $$\left(\begin{array}{ll}
        c_{11} & c_{12} \\
        c_{21} & c_{22}
        \end{array}\right)$$
    这个矩阵称为随机变量$(X_1,X_2)$的{\heiti 协方差矩阵}。

    设$n$维随机变量$(X_1,X_2,\cdots,X_n)$的二阶混合中心矩$c_{ij}=\mathrm{Cov}(X_i,X_j)=E\{[X_i-E(X_i)][X_j-E(X_j)]\},i,j=1,2,\cdots,n$
    都存在,则称矩阵
    $$\boldsymbol{C}=\left(\begin{array}{llll}
        c_{11} & c_{12} & \cdots & c_{1n}\\
        c_{21} & c_{22} & \cdots & c_{2n}\\
        \vdots & \vdots &        & \vdots \\
        c_{n1} & c_{n2} & \cdots & c_{nn}
        \end{array}\right)$$
    为$n$维随机变量$(X_1,X_2,\cdots,X_n)$的{\heiti 协方差矩阵},该矩阵是一个对称矩阵。
\end{definition}

