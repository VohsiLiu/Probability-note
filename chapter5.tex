\section{大数定律及中心极限定理}
\subsection{大数定律}

\begin{theorem}[弱大数定理(辛钦大数定理)]
    设$X_1,X_2,\cdots$是相互独立的,服从同一分布的随机变量序列,且具有数学期望$E(X_k)=\mu (k=1,2,\cdots)$。
    作前$n$个变量的算数平均$\displaystyle{\frac{1}{n} \sum_{k=1}^n X_k }$,则对于任意$\varepsilon>0$,有
    $$\lim_{n\to \infty} P \left\{ \left\lvert  \frac{1}{n} \sum_{k=1}^n X_k-\mu \right\rvert <\varepsilon \right\} =1 $$
\end{theorem}

\begin{definition}
    设$Y_1,Y_2,\cdots,Y_n,\cdots$是一个随机变量序列,$a$是一个常数。若对于任意正数$\varepsilon$,有
    $$\lim_{n\to \infty} P \left\{ \left\lvert Y_n-a \right\rvert <\varepsilon \right\} =1 $$
    则称序列$Y_1,Y_2,\cdots,Y_n,\cdots${\heiti 依概率收敛于$a$},记为
    $$Y_n \stackrel{P}{\longrightarrow} a$$
\end{definition}

\begin{theorem}[弱大数定理(辛钦大数定理)]
    设随机变量$X_1,X_2,\cdots$相互独立,服从同一分布,且具有数学期望$E(X_k)=\mu (k=1,2,\cdots)$,则
    序列$\displaystyle{\overline{X}=\frac{1}{n}\sum _{k=1}^n X_k}$依概率收敛于$\mu$,即$\overline{X} \stackrel{P}{\longrightarrow} a$
\end{theorem}

\begin{theorem}[伯努利大数定理]
    设$f_A$是$n$次独立重复试验中事件$A$发生的次数,$p$是事件$A$在每次试验中发生的概率,则对于任意正数$\varepsilon>0$,有
    $$\lim_{n\to \infty} P \left\{ \left\lvert  \frac{f_A}{n}-p  \right\rvert <\varepsilon \right\} =1 $$或
    $$\lim_{n\to \infty} P \left\{ \left\lvert  \frac{f_A}{n}-p  \right\rvert \geq \varepsilon \right\} =0 $$
\end{theorem}

\subsection{中心极限定理}
\begin{theorem}[独立同分布的中心极限定理]
    设随机变量$X_1,X_2,\cdots,X_n,\cdots$相互独立,服从同一分布,且具有数学期望和方差:$E(X_k)=\mu,D(X_k)=\sigma^2>0(k=1,2,\cdots),$则随机变量
    之和$\displaystyle{\sum _{k=1}^n X_k}$的标准化变量
    $$Y_n=\frac{\sum\limits _{k=1}^n X_k-E(\sum\limits _{k=1} ^n X_k)}{\sqrt{D(\sum\limits_{k=1}^n X_k)}}=\frac{\sum\limits_{k=1}^n X_k-n\mu}{\sqrt{n}\sigma}$$
    的分布函数$F_n(x)$对于任意$x$满足
    $$\lim _{n\to\infty} F_n(x)=\lim _{n\to \infty}P\left\{ \frac{\sum\limits_{k=1}^n X_k-n\mu}{\sqrt{n}\sigma}\leq x\right\}=\int_{-\infty}^{x} \frac{1}{\sqrt{2\pi}} e^{-t^2/2}\,dt=\varPhi (x)  $$
\end{theorem}
\begin{remark}
    该定理表明,均值为$\mu$,方差为$\sigma^2>0$的独立同分布的随机变量$X_1,X_2,\cdots,X_n$之和$\displaystyle{\sum _{k=1}^n X_k}$的标准化变量,当$n$充分大时,有
    $$\frac{\sum\limits_{k=1}^n X_k -n\mu}{\sqrt{n}\sigma}\sim N(0,1) $$
    也可改写为$$\frac{\overline{X}-\mu}{\sigma/\sqrt{n}} \sim N(0,1) \quad \mbox{或} \quad \overline{X}\sim N(\mu,\sigma^2/n)$$
\end{remark}

\begin{theorem}[李雅普诺夫(Lyapunov)定理]
    设随机变量$X_1,X_2,\cdots,X_n,\cdots$相互独立,它们具有数学期望和方差$E(X_k)=\mu, D(X_k)=\sigma_k^2>0,k=1,2,\cdots$,记$\displaystyle{B_n^2=\sum_{k=1}^n\sigma_k^2}$。
    若存在整数$\delta$,使得当$n\to \infty$时,
    $$\frac{1}{B_n^{2+\delta}}\sum_{k=1}^n E\{{|X_k-\mu_k|}^{2+\delta}\} \to 0,$$
    则随机变量之和$\displaystyle{\sum _{k=1}^n X_k}$的标准化变量
    $$Z_n=\frac{\sum\limits _{k=1}^n X_k-E(\sum\limits _{k=1} ^n X_k)}{\sqrt{D(\sum\limits_{k=1}^n X_k)}}=\frac{\sum\limits_{k=1}^n X_k-\sum\limits_{k=1}^n \mu_k}{B_n}$$
    的分布函数$F_n(x)$对任意$x$,满足
    $$\lim_{n\to\infty} F_n(x)=\lim_{n\to\infty} P\left\{  \frac{\sum\limits_{k=1}^n X_k-\sum\limits_{k=1}^n \mu_k}{B_n}\leq x  \right\} =\int_{-\infty}^{x} \frac{1}{\sqrt{2\pi}} e^{-t^2/2}\,dt=\varPhi (x) $$
\end{theorem}

\begin{remark}
    该定理表明,在定理的条件下,随机变量 $$Z_n=\frac{\sum\limits_{k=1}^n X_k-\sum\limits_{k=1}^n \mu_k}{B_n}$$
    当$n$很大时,近似地服从正态分布$N(0,1)$。由此,当$n$很大时,$\displaystyle{\sum_{k=1}^n X_k=B_nZ_n+\sum_{k=1}^n \mu_k}$
    近似地服从正态分布$\displaystyle{N(\sum_{k=1}^n\mu_k,B_n^2)}$。这也就是说,无论各个随机变量$X_k(k=1,2,\cdots)$服从什么分布,
    只要满足定理的条件,那么它们的和$\displaystyle{\sum_{k=1}^nX_k}$当$n$很大时,就近似地服从正态分布。
\end{remark}

\begin{theorem}[棣莫弗-拉普拉斯(De Moivre-Laplace)定理]
    设随机变量$\eta_n(n=1,2,\cdots)$服从参数为$n,p(0<p<1)$的二项分布,则对于任意$x$,有
    $$\lim_{n\to\infty}P\left\{   \frac{\eta_n-np}{\sqrt{np(1-p)}}\leq x\right\} =\int_{-\infty}^{x} \frac{1}{\sqrt{2\pi}} e^{-t^2/2}\,dt=\varPhi (x) $$
\end{theorem}

\begin{remark}
    该定理表明,正态分布时二项分布的极限分布。当$n$充分大时,可以用该定理计算二项分布的概率。
\end{remark}

