\section{样本及抽样分布}
\subsection{随机样本}

\begin{definition}
    设$X$是具有分布函数$F$的随机变量,若$X_1,X_2,\cdots,X_n$是具有同一分布函数$F$的、相互独立的随机变量,则称$X_1,X_2,\cdots,X_n$为从分布
    函数$F$(或总体$F$、或总体$X$)得到的{\heiti 容量为$n$的简单随机样本},简称{\heiti 样本},它们的观察值$x_1,x_2,\cdots,x_n$称为
    {\heiti 样本值},又称为$X$的$n$个{\heiti 独立的观察值}。

    由定义得:若$X_1,X_2,\cdots,X_n$为$F$的一个样本,则$X_1,X_2,\cdots,X_n$相互独立,且它们的分布函数都是$F$,所以$(X_1,X_2,\cdots,X_n)$
    的分布函数为
    $$F^\ast(x_1,x_2,\cdots,x_n)=\prod _{i=1}^n F(x_i)$$
    又若$X$具有概率密度$f$,则$(X_1,X_2,\cdots,X_n)$的概率密度为
    $$f^\ast(x_1,x_2,\cdots,x_n)=\prod _{i=1}^n f(x_i)$$
\end{definition}

\subsection{直方图和箱线图}
直方图和箱线图的定义及画法略。

\subsection{抽样分布}
\begin{definition}
    \begin{definition}
        设$X_1,X_2,\cdots,X_n$是来自总体$X$的一个样本,$g(X_1,X_2,\cdots,X_n)$是
        $X_1,X_2,\cdots,X_n$的函数,若$g$中不含未知参数,则称$g(X_1,X_2,\cdots,X_n)$
        是一{\heiti 统计量}。
    \end{definition}
\end{definition}