\section{多维随机变量及其分布}

\subsection{二维随机变量}
\begin{definition}[联合分布函数]
    设$(X,Y)$是二维随机变量,对于任意实数$x,y$,二元函数
    $$F(x,y)=P\{(X\leq x)\cap(Y\leq y)\} \equiv P\{X\leq x,Y\leq y\}$$
    称为二维随机变量$(X,Y)$的{\heiti 分布函数},或称为随机变量$X$和$Y$的{\heiti 联合分布函数}。

    如果将二维随机变量$(X,Y)$看成是平面上随机点的坐标,那么分布函数$F(x,y)$在$(x,y)$处的函数值就是随机点
    $(X,Y)$落在以点$(x,y)$为顶点而位于该点左下方的无穷矩形区域内的概率。
    \\所以,随机点$(X,Y)$落在矩形区域$\{(x,y)|x_1<x\leq x_2,y_1<y\leq y_2\}$的概率为
    $$P\{x_1<x\leq x_2,y_1<y\leq y_2\}=F(x_2,y_2)-F(x_2,y_1)+F(x_1,y_1)-F(x_1,y_2)$$
\end{definition}

\begin{definition}[联合概率密度]
    对于二维随机变量$(X,Y)$的分布函数$F(x,y)$,如果存在非负可积函数$f(x,y)$使对于任意$x,y$有
    $$F(x,y)=\int_{-\infty}^{y}\int_{-\infty}^x f(u,v)\,dudv$$
    则称$(X,Y)$是{\heiti 连续型的二维随机变量},函数$f(x,y)$称为二维随机变量$(X,Y)$的{\heiti 概率密度},
    或称为随机变量$X$和$Y$的{\heiti 联合概率密度}。
\end{definition}

\subsection{边缘分布}
\begin{definition}[边缘分布函数]
    二维随机变量$(X,Y)$作为一个整体,具有分布函数$F(x,y)$,而$X$和$Y$都是随机变量,各自也有分布函数,将他们分别记为
    $F_X(x),F_Y(y)$,依次称为二维随机变量$(X,Y)$关于$X$和关于$Y$的{\heiti 边缘分布函数},且
    $$F_X(x)=F(x,\infty),F_Y(y)=F(\infty,y)$$
\end{definition}

\begin{definition}[边缘分布律]
    记
    $$p_{i\cdot}=\sum\limits_{j=1}^\infty p_{ij}=P\{X=x_i\},\quad i=1,2,\cdots,$$
    $$p_{\cdot j}=\sum\limits_{i=1}^\infty p_{ij}=P\{Y=y_j\},\quad j=1,2,\cdots,$$
    分别称$p_{i\cdot}(i=1,2,\cdots)$和$p_{\cdot j}(j=1,2,\cdots)$为$(X,Y)$关于$X$和$Y$的{\heiti 边缘分布律}
\end{definition}

\begin{definition}[边缘概率密度]
    记
    $$f_X(x)=\int_{-\infty}^{\infty}f(x,y)\,dy$$
    $$f_Y(y)=\int_{-\infty}^{\infty}f(x,y)\,dx$$
    分别称$f_X(x),f_Y(y)$为$(X,Y)$关于$X$和$Y$的边缘概率密度。
\end{definition}

\subsection{条件分布}
\begin{definition}[条件分布律]
    设$(X,Y)$是二维离散型随机变量,对于固定的$j$,若$P\{Y=y_j\}>0$,则称
    $$P\{X=x_i|Y=y_j\}=\frac{P\{X=x_i,Y=y_j\}}{P\{Y=y_j\}}=\frac{p_{ij}}{p_{\cdot j}},i=1,2,\cdots$$
    为在$Y=y_j$条件下随机变量$X$的条件分布律

    同样,对于固定的$i$,若$P\{X=x_i\}>0$,则称
    $$P\{Y=y_j|X=x_i\}=\frac{P\{X=x_i,Y=y_j\}}{P\{X=X_i\}}=\frac{p_{ij}}{p_{i\cdot }},j=1,2,\cdots$$
    为在$X=x_i$条件下随机变量$Y$的条件分布律
\end{definition}

\begin{definition}[条件概率密度]
    设二维随机变量$(X,Y)$的概率密度为$f(x,y)$,$(X,Y)$关于$Y$的边缘概率密度为$f_Y(y)$.若对于固定的$y,f_Y(y)>0$,则称$\frac{f(x,y)}{f_Y(y)}$
    为在$Y=y$的条件下$X$的条件概率密度,记为
    $$f_{X|Y}(x|y)=\frac{f(x,y)}{f_Y(y)}$$
    称$\displaystyle{\int_{-\infty}^x f_{X|Y}(x|y)\,dx=\int_{-\infty}^x \frac{f(x,y)}{f_Y(y)}\,dx}$为在$Y=y$的条件下$X$的条件分布函数,记为
    $P\{X\leq x|Y=y\}$或$F_{X|Y}(x|y)$,即
    $$F_{X|Y}(x|y)=P\{X\leq x|Y=y\}=\int_{-\infty}^x \frac{f(x,y)}{f_Y(y)}\,dx$$

    类似地,可以定义$\displaystyle{f_{Y|X}(y|x)=\frac{f(x,y)}{f_X(x)}}$和$\displaystyle{F_{Y|X}(y|x)=\int_{-\infty}^y \frac{f(x,y)}{f_X(x)}\,dy}$
\end{definition}

\subsection{相互独立的随机变量}
\begin{definition}[相互独立]
    若对于所有$x,y$,满足下列条件之一,则称随机变量$X$和$Y$是相互独立的:
    \begin{enumerate}
        \item 设$F(x,y)$及$F_X(x),F_Y(y)$分别是二维随机变量$(X,Y)$的分布函数及边缘分布函数,有
                $$F(x,y)=F_X(x)F_Y(y)$$
        \item 设$f(x,y),f_X(X),f_Y(y)$分别为$(X,Y)$的概率密度和边缘概率密度,有
                $$f(x,y)=f_X(x)f_Y(y)$$
        \item 当$X,Y$是离散型随机变量是,对于$(X,Y)$的所有可能取值$(x_i,y_i)$,有
                $$P\{X=x_i,Y=y_i\}=P\{X=x_i\}P\{Y=y_i\}$$
    \end{enumerate}
\end{definition}

\begin{theorem}
    设$(X_1,X_2,\cdots,X_m)$和$(Y_1,Y_2,\cdots,Y_n)$相互独立,则$X_i(i=1,2,\cdots,m)$和$Y_j(j=1,2,\cdots,n)$
    相互独立。又若$h,g$是连续函数,则$h(X_1,X_2,\cdots,X_m)$和$g(Y_1,Y_2,\cdots,Y_n)$相互独立。
\end{theorem}

\subsection{两个随机变量的函数的分布}
\begin{definition}[$Z=X+Y$分布]
    设$(X,Y)$是二维连续型随机变量,它具有概率密度$f(x,y)$,则$Z=X+Y$仍为连续型随机变量,其概率密度为
    \begin{equation}
        f_{X+Y}(z)=\int_{-\infty}^\infty f(z-y,y)\,dy,\label{1.1}
    \end{equation}
    或
    \begin{equation}
        f_{X+Y}(z)=\int_{-\infty}^\infty f(x,z-x)\,dx.\label{1.2}
    \end{equation}
    
    又若$X$和$Y$相互独立,设$(X,Y)$关于$X,Y$的边缘密度分别为$f_X(x),f_Y(y)$,则\eqref{1.1},\eqref{1.2}分别可化为
    $$f_{X+Y}(z)=\int_{-\infty}^\infty f_X(z-y)f_Y(y)\,dy$$
    和$$f_{X+Y}(z)=\int_{-\infty}^\infty f_X(x)f_Y(z-x)\,dx$$
    这两个公式称为$f_X$和$f_Y$的{\heiti 卷积公式},记为$f_X \ast f_Y$,即
    $$f_X \ast f_Y=\int_{-\infty}^\infty f_X(z-y)f_Y(y)\,dy= \int_{-\infty}^\infty f_X(x)f_Y(z-x)\,dx$$
\end{definition}

\begin{theorem}
    有限个相互独立的正态随机变量的线性组合仍然服从正态分布
\end{theorem}

\begin{definition}[${Z=\dfrac{Y}{X}}$分布、$Z=XY$的分布]
    设$(X,Y)$是二维连续型随机变量,它具有概率密度$f(x,y)$,则${Z=\dfrac{Y}{X}},Z=XY$仍为连续型随机变量,其概率密度分别为
    \begin{equation}
        f_{Y/X}(z)=\int_{-\infty}^\infty |x|f(x,xz)\,dx;\label{2.1}
    \end{equation}
    \begin{equation}
        f_{XY}(z)=\int_{-\infty}^\infty \frac{1}{|x|}f(x,\frac{z}{x})\,dx.\label{2.2}
    \end{equation}
    又若$X$和$Y$相互独立,设$(X,Y)$关于$X,Y$的边缘密度分别为$f_X(x),f_Y(y)$,则\eqref{2.1}可化为
        $$f_{Y/X}(z)=\int_{-\infty}^\infty |x|f_X(x)f_Y(xz)\,dx$$
    \eqref{2.2}可化为
        $$f_{XY}(z)=\int_{-\infty}^\infty \frac{1}{|x|}f_X(x)f_Y(\frac{z}{x})\,dx$$
\end{definition}

\begin{definition}[$M=\max\{X,Y\}$分布及$N=\min\{X,Y\}$分布]
    设$X,Y$是两个相互独立的随机变量,它们的分布函数分别为$F_X(x)$和$F_Y(y)$,可得$M=\max\{X,Y\}$的分布函数为
    $$F_{\max}(z)=F_X(z)F_Y(z)$$
    类似地,可得到$N=\min\{X,Y\}$的分布函数为
    $$F_{\min}(z)=1-[1-F_X(z)][1-F_Y(z)]$$

    以上结果容易推广到$n$个相互独立的随机变量的情况。设$X_1,X_2,\cdots,X_n$是$n$个相互独立的随机变量,它们的分布函数分别为
    $F_{X_i}(x_i)(i=1,2,\cdots,n)$,则$M=\max\{X_1,X_2,\cdots,X_n\}$及$N=\min\{X_1,X_2,\cdots,X_n\}$的分布函数分别为
    $$F_{\max}(z)=F_{X_1}(z)F_{X_2}(z)\cdots F_{X_n}(z)$$
    $$F_{\min}(z)=1-[1-F_{X_1}(z)][1-F_{X_2}(z)]\cdots [1-F_{X_n}(z)]$$

    特别地,当$X_1,X_2,\cdots,X_n$相互独立且具有相同分布函数$F(x)$时有
    $$F_{\max}(z)=[F(z)]^n$$
    $$F_{\min}(z)=1-[1-F(z)]^n$$
\end{definition}
