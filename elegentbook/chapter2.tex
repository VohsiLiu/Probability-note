\chapter{随机变量及其分布}

\section{随机变量}
\begin{definition}[随机变量]
  设随机试验的样本空间为$S=\{e\}$,$X=X(e)$是定义在样本空间$S$上的实值单值函数,称$X=X(e)$为随机变量
\end{definition}

\section{离散型随机变量及其分布律}
\begin{definition}[离散型随机变量]
    全部可能取到的值是有限个或可列无限多个的随机变量,称为离散型随机变量。
\end{definition}

\begin{definition}[($0-1$)分布]
    设随机变量$X$只能取$0,1$两个值,它的分布律是
    $$P\{X=k\}=p^k(1-p)^{1-k},k=0,1 \quad (0<p<1)$$
\end{definition}

\begin{definition}[伯努利试验]
    设试验$E$只有两个可能结果:$A$和$\overline{A} $,则称$E$为伯努利(Bernoulli)试验.设$P(A)=p(0<p<1)$,
    此时$P(\overline{A})=1-p$,将$E$独立重复进行$n$次,则称这一串重复的独立试验为$n$重伯努利试验。
\end{definition}

\begin{definition}[二项分布]
    以$X$表示$n$重伯努利试验中事件$A$发生的次数,事件$A$在指定的$k(0\leq k\leq n)$次试验中发生的概率为
    $$P\{X=k\}=\binom{n}{k}p^k(1-p)^{n-k},k=0,1,2,\cdots,n$$
    随机变量$X$服从参数为$n,p$的二项分布,并记为$X\sim b(n,p)$. 
\end{definition}

\begin{definition}[泊松分布]
    设随机变量$X$所有可能取的值为$0,1,2,\cdots$,而取各个值的概率为
    $$P\{X=k\}=\frac{\lambda^ke^{-\lambda}}{k!},k=0,1,2,\cdots,$$
    其中$\lambda>0$是常数。则称$X$服从参数为$\lambda$的泊松分布,记为$X\sim \pi(\lambda) $
\end{definition}

\begin{theorem}[泊松定理]
    设$\lambda>0$是一个常数,$n$是任意正整数,设$np_n=\lambda$,则对于任一固定的非负整数$k$,有
    $$\lim_{n \to \infty}\binom{n}{k}{p_n}^k(1-p_n)^{n-k}=\frac{\lambda^ke^{-\lambda}}{k!}$$   
\end{theorem}

\section{随机变量的分布函数}
\begin{definition}[随机变量的分布函数]
    设$X$是一个随机变量,$x$是任意实数,函数
    $$F(X)=P\{X\leq x\},-\infty<x<\infty$$
    称为$X$的分布函数
\end{definition}

\begin{theorem}[分布函数性质]
      $\quad$  

    \begin{enumerate}
        \item $F(x)$是一个不减函数
        \item $0\leq x\leq 1$,且$F(-\infty)=\lim\limits_{x \to -\infty}  F(x)=0,F(\infty)=\lim\limits_{x \to \infty}  F(x)=1$
        \item $F(x+0)=F(x)$,即$F(x)$是右连续的
    \end{enumerate}

    反之,具备上述三条性质的函数必是某个随机变量的分布函数。
\end{theorem}

\section{连续型随机变量及其概率密度}
\begin{definition}[连续型随机变量、概率密度]
    如果对于随机变量$X$的分布函数$F(x)$,存在非负可积函数$f(x)$,使对于任意实数$x$有
    $$F(x)=\int_{-\infty}^{x} f(t) \,dt $$
    则称$X$为{\heiti 连续型随机变量},$f(x)$称为$X$的概率密度函数,简称{\heiti 概率密度}。
\end{definition}

\begin{definition}[均匀分布]
    若连续型随机变量$X$具有概率密度
    $$ f(x)=\left\{
    \begin{array}{lll}
    \frac{1}{b-a}, &  & a<x<b\\
    0,&  &  \mbox{其他} \\
    \end{array}\right. $$
    则称$X$在区间$(a,b)$上服从均匀分布,记为$X\sim U(a,b)$\\
    $X$的分布函数为
    $$ F(x)=\left\{
        \begin{array}{lll}
        0, &  & x<a\\
        \frac{x-a}{b-a}, &  &a\leq x<b\\
        1,&  &  x\geq b \\
        \end{array}\right. $$
\end{definition}

\begin{definition}[指数分布]
    若连续型随机变量$X$的概率密度为
    $$ f(x)=\left\{
        \begin{array}{lll}
        \frac{1}{\theta}e^{-\frac{x}{\theta}}, &  & x>0\\
        0,&  &  \mbox{其他} \\
        \end{array}\right. $$
    其中$\theta>0$为常数,则称$X$服从参数为$\theta$的指数分布\\
    $X$的分布函数为
    $$ F(x)=\left\{
        \begin{array}{lll}
        1-e^{-\frac{x}{\theta}}, &  &x>0\\
        0,&  &  \mbox{其他} \\
        \end{array}\right. $$
\end{definition}

\begin{definition}[正态分布]
    若连续型随机变量$X$的概率密度为
    $$f(x)=\frac{1}{\sqrt{2\pi}\sigma}e^{-\frac{{(x-\mu)}^2}{2\sigma^2}}, \quad -\infty<x<\infty$$
    其中$\mu,\sigma(\sigma>0)$为常数,则称$X$服从参数为$\mu,\sigma$的正态分布或高斯分布,记为$X\sim N(\mu,\sigma^2)$\\
    $X$的分布函数为
    $$F(x)=\frac{1}{\sqrt{2\pi}\sigma}\int_{-\infty}^{x} e^{-\frac{{(t-\mu)}^2}{2\sigma^2}} \,dt $$

    特别地,当$\mu=0,\sigma=1$时,称随机变量$X$服从{\heiti 标准正态分布},其概率密度和分布函数分别用$\varphi(x),\varPhi(x)$表示,有
    $$ \varphi(x)=\frac{1}{\sqrt{2\pi}}e^{-\frac{t^2}{2}}$$
    $$ \varPhi(x)=\frac{1}{\sqrt{2\pi}}\int_{-\infty}^x e^{-\frac{t^2}{2}} \,dt$$

    对于一般的正态分布,只需通过一个线性变换就能化为标准正态分布:
    $$\mbox{若}X\sim N(\mu,\sigma^2),\mbox{则}Z=\frac{X-\mu}{\sigma}\sim N(0,1)$$
\end{definition}

\section{随机变量的函数的分布}
\begin{theorem}
    设随机变量$X$具有概率密度$f_x(x),-\infty<x<\infty$,又设函数$g(x)$处处可导且恒有$g'(x)>0$(或恒有$g'(x)<0$,
    则$Y=g(X)$是连续型随机变量,其概率密度为
    $$f_Y(y)=\left\{
        \begin{array}{lll}
        f_x[h(y)]|h'(y)|, &  &\alpha<y<\beta\\
        0,&  &  \mbox{其他} \\
        \end{array}\right. $$
    其中$\alpha=\min\{g(-\infty),g(\infty)\},\beta=\max\{g(-\infty),g(\infty)\}$,$h(y)$是$g(x)$的反函数
\end{theorem}